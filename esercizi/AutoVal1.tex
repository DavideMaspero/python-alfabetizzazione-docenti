%!TEX program = xelatex
\documentclass[a4]{article}
\usepackage{fontspec, graphicx}
\usepackage{xunicode}
%	\usepackage{xltxtra}
\usepackage{polyglossia}
%	\setmainlanguage[]{italian}
\defaultfontfeatures{Mapping=tex-text} % converts LaTeX specials (``quotes'' --- dashes etc.) to unicode
\setmainfont[Ligatures=TeX]{Gentium Basic}
\setsansfont[Ligatures=TeX,Scale=MatchLowercase]{Gentium Basic}
\setmonofont[Scale=MatchLowercase]{Inconsolata}
\usepackage[xetex]{media9}



\graphicspath{{/Users/dario/Works/images/},{/Users/dariopescini/Works/images/}}
%%%%%%%%%%%%%%%%%%%%%%%%%
\begin{document}

{\noindent \Large{\bf Si risolvano i seguenti esercizi esplicitando i seguenti passaggi:}
\begin{itemize}
	\item definizione di input ed output della procedura
	\item disegno del corrispondente diagramma di flusso
	\item implementazione in Python.
\end{itemize}

\vspace{.5cm}
\begin{enumerate}
	\item Scrivere un programma che stabilisca a quale quadrante o asse cartesiano, appartiene un punto di coordinate $x$ ed $y$ inserite dall'utente.\\ \includegraphics[width=.25\textwidth]{InfStatPyQuadranti}\\[.25cm]
	\item Scrivere un programma che stabilisca se tre numeri, inseriti dall'utente, possono essere i tre lati di un triangolo. Si sfrutti la propietà che ciascun lato deve essere minore della somma degli altri due.\\[.25cm]
	\item Si scriva un programma che computi il costo della sosta (durata massima 24h, espresso in minuti) in un parcheggio a pagamento che applichi le seguenti tariffe:
	\begin{itemize}
		\item Gratuita per i primi  30 min.
		\item dal 31° min. al 120° min.    euro 1,00
		\item dal 121° min. fino alla 6° ora  euro 0,50/ora
		\item oltre la 6° ora  euro 1,00/ora fino ad un massimo di euro 16,00
	\end{itemize}
\end{enumerate}
}
\end{document}
